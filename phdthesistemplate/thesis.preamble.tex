\usepackage{graphicx}
\usepackage{verbatim}
\usepackage{latexsym}
\usepackage{mathchars}
\usepackage{setspace}
\usepackage[toc,page]{appendix}
\usepackage{float}
% \usepackage{subfigure}
\usepackage{subcaption}
\usepackage{url}
\usepackage{hyperref}
\hypersetup{
     colorlinks=true,
     linkcolor=black,
     filecolor=black,
     citecolor=black,
     urlcolor=blue,
}
% \usepackage[colorlinks=true]{hyperref}
     
\usepackage{tabularx}
\usepackage{booktabs}
\usepackage{multirow}
\usepackage[normalem]{ulem}
\usepackage{pdfpages}

\usepackage[htt]{hyphenat}

% \usepackage{apacite}
% \usepackage{natbib}
% \usepackage[sectionbib]{natbib}
% \bibliographystyle{apalike}
\usepackage{apalike}
\usepackage{natbib}

\usepackage[acronym]{glossaries}

% for H1. H2. enumeration
\usepackage{enumitem}

\usepackage{amsmath}
\usepackage{changepage}

\usepackage{acro}

% for C# code formatting
\usepackage{color}
\usepackage{listings}
\usepackage{spverbatim}
\lstloadlanguages{% Check Dokumentation for further languages ...
C,
C++,
csh,
Java
}

\definecolor{red}{rgb}{0.6,0,0} % for strings
\definecolor{blue}{rgb}{0,0,0.6}
\definecolor{green}{rgb}{0,0.8,0}
\definecolor{cyan}{rgb}{0.0,0.6,0.6}
\lstset{
basicstyle=\small\ttfamily, 
stringstyle=\color{black}\ttfamily, 
showstringspaces=false,
showspaces=false,
xleftmargin=17pt,
breaklines=true,
}
\usepackage{caption}

% for table of contents to show Chapter
\usepackage{tocloft,calc}
\renewcommand{\cftchappresnum}{Chapter }
\AtBeginDocument{\addtolength\cftchapnumwidth{\widthof{Chapter}}}

% for table of contents to show Appendix
\renewcommand{\cftchappresnum}{\chaptername\space}
\setlength{\cftchapnumwidth}{\widthof{Appendix}}
\makeatletter
\g@addto@macro\appendix{%
  \addtocontents{toc}{%
    \protect\renewcommand{\protect\cftchappresnum}{\appendixname\space}%
  }%
}

% mini table of content
\usepackage{minitoc}
\setlength{\mtcindent}{24pt}
\renewcommand{\mtcoffset}{0pt}
\mtcsetoffset{minitoc}{0pt} \setlength{\mtcskipamount}{\bigskipamount}
\mtcsetdepth{minitoc}{2} \mtcsetfont{minitoc}{*}{\small\rmfamily\upshape\mdseries} \mtcsetfont{minitoc}{section}{\small\rmfamily\upshape\bfseries}
\setcounter{mtc}{9}

\usepackage{fancyhdr}
% \fancypagestyle{plain}{%  the preset of fancyhdr 
%     \fancyhf{} % clear all header and footer fields
%     \fancyhead[C]{\textbf{\rightmark}}
%     % \fancyhead[R]{\textbf{\rightmark}}
%     % \pagenumbering{arabic}
%     % \fancyfoot[R]{\thepage}
%     \renewcommand{\headrulewidth}{0pt}
%     \renewcommand{\footrulewidth}{0pt}}
    
\fancypagestyle{front}{% style for TOC, LOF, LOT
  \fancyhf{}
  \renewcommand{\headrulewidth}{0pt}
  \cfoot{\thepage}
}
\fancypagestyle{main}{% style for the mainmatter
  \fancyhf{}
  \fancyhead[C]{\slshape \rightmark}
  \fancyfoot[C]{\thepage}
  \renewcommand\headrulewidth{.4pt}
}
\makeatletter
  \newcommand\frontpagestyle{\cleardoublepage\pagestyle{front}\let\ps@plain\ps@front}
  \newcommand\mainpagestyle{\cleardoublepage\pagestyle{main}\let\ps@plain\ps@main}
\makeatother

\input{blocked.sty}
% \input{uhead.sty}
\input{boxit.sty}
% Created by Yue Li, June 2017

\pagestyle{empty}

\setlength{\parskip}{0em}
\setlength{\parindent}{0em}

\makeatletter  %to avoid error messages generated by "\@". Makes Latex treat "@" like a letter

\linespread{1.5}
\def\submitdate#1{\gdef\@submitdate{#1}}
\def\degree#1{\gdef\@degree{#1}}
\def\studentid#1{\gdef\@studentid{#1}}
\def\supervisor#1{\gdef\@supervisor{#1}}

\def\maketitle{
    
  \begin{titlepage}{
    
    \centering{
    \includegraphics[width=0.5\columnwidth]{images/nottingham-logo.png}} \par
    % \textbf{International Doctoral Innovation Centre}
    
    \vskip 1in \par 
    \LARGE {\bf \@title}
    
    \vskip 0.5in \par
    \normalsize {Thesis submitted to the University of Nottingham for the degree of \\\bf{\@degree}, \@submitdate.}

  }
  \vskip 0.3in \par
  \large {\bf \@author} \par
  \large {\bf \@studentid}
  
  \vskip 0.3in \par
  \normalsize {Supervised by} \par
  \normalsize {\textbf{\@supervisor}}
  \vskip 0.3in \par
%   \normalsize { School of Computer Science \par
%   University of Nottingham}

  \vskip 0.5in \par
%   \normalsize {I hereby declare that this dissertation is all my own work, except as indicated in the text: }

  \vskip 0.5in 
  \normalsize {Signature \underline{\hspace{1.5in}}}
  
  \vskip 0.1in
  \normalsize {Date \underline{\hspace{0.5in}} / \underline{\hspace{0.5in}} / \underline{\hspace{0.5in}}}
  
  %%%%%%%%%%
  %*Only include this sentence below if you do have all necessary rights and consents. For example, if you have including photographs or images from the web or from other papers or documents then you need to obtain explicit consent from the original copyright owner. If in doubt, delete this sentence. See Copyright Information: http://eprints.nottingham.ac.uk/copyrightinfo.html for more details.
  %%%%%%%%%%
  
%   \vskip 0.4in \par
%   \normalsize {I hereby declare that I have all necessary rights and consents to publicly distribute this dissertation via the University of Nottingham's e-dissertation archive.}

  %%%%%%%%%%
  %Only include this sentence below if there is some reason why your dissertation should not be accessible for some period of time, for example if it contains information which is commercially sensitive or might compromise an Intellectual Property claim. If included, fill in the date from which access should be allowed.
  %%%%%%%%%%

%   \vskip 0.4in \par
%   \normalsize {Public access to this dissertation is restricted until: DD/MM/YYYY}
  
  \end{titlepage}
}

\def\titlepage{
  \newpage
  \centering
  \linespread{1}
  \normalsize
  \vbox to \vsize\bgroup\vbox to 9in\bgroup
}
\def\endtitlepage{
  \par
  \kern 0pt
  \egroup
  \vss
  \egroup
  \cleardoublepage
}

\def\abstract{
  \begin{center}{
    \large\bf Abstract}
  \end{center}
  \small
  %\def\baselinestretch{1.5}
  \linespread{1.5}
  \normalsize
}
\def\endabstract{
  \par
}

\newenvironment{acknowledgements}{
  \cleardoublepage
  \begin{center}{
    \large \bf Acknowledgements}
  \end{center}
  \small
  \linespread{1.5}
  \normalsize
}{\cleardoublepage}
\def\endacknowledgements{
  \par
}

\def\preface{
    \pagenumbering{roman}
    \pagestyle{plain}
    \doublespacing
}

\setlength\cftbeforechapskip{1pt}

\def\body{
	\setcounter{tocdepth}{1}
	\setcounter{minitocdepth}{2}
	
	\dominitoc 
	\dominilof 
	\dominilot
	
    \tableofcontents
    \pagestyle{plain}
    
    \cleardoublepage
    \listoftables
    \addstarredchapter{List of Tables}
    \chaptermark{List of Tables}
    \pagestyle{plain}
    
    \cleardoublepage
    \addstarredchapter{List of Figures}
    \chaptermark{List of Figures}
    \listoffigures
    \pagestyle{plain}
    
    \cleardoublepage
    \printnoidxglossary[type=\acronymtype,style=index,nonumberlist,title=Abbreviations, nogroupskip=true]
    \printnoidxglossary[title=Nomenclature]
    \addstarredchapter{Abbreviations}
    \chaptermark{Abbreviations}
    \pagestyle{plain}
    
    \cleardoublepage
    \pagenumbering{arabic}
    \doublespacing
    
    \adjustmtc
}

\makeatother  %to avoid error messages generated by "\@". Makes Latex treat "@" like a letter

\newcommand{\ipc}{{\sf ipc}}
\newcommand{\Prob}{\bbbp}
\newcommand{\Real}{\bbbr}
\newcommand{\real}{\Real}
\newcommand{\Int}{\bbbz}
\newcommand{\Nat}{\bbbn}

\newcommand{\NN}{{\sf I\kern-0.14emN}}   % Natural numbers
\newcommand{\ZZ}{{\sf Z\kern-0.45emZ}}   % Integers
\newcommand{\QQQ}{{\sf C\kern-0.48emQ}}   % Rational numbers
\newcommand{\RR}{{\sf I\kern-0.14emR}}   % Real numbers
\newcommand{\KK}{{\cal K}}
\newcommand{\OO}{{\cal O}}
\newcommand{\AAA}{{\bf A}}
\newcommand{\HH}{{\bf H}}
\newcommand{\II}{{\bf I}}
\newcommand{\LL}{{\bf L}}
\newcommand{\PP}{{\bf P}}
\newcommand{\PPprime}{{\bf P'}}
\newcommand{\QQ}{{\bf Q}}
\newcommand{\UU}{{\bf U}}
\newcommand{\UUprime}{{\bf U'}}
\newcommand{\zzero}{{\bf 0}}
\newcommand{\ppi}{\mbox{\boldmath $\pi$}}
\newcommand{\aalph}{\mbox{\boldmath $\alpha$}}
\newcommand{\bb}{{\bf b}}
\newcommand{\ee}{{\bf e}}
\newcommand{\mmu}{\mbox{\boldmath $\mu$}}
\newcommand{\vv}{{\bf v}}
\newcommand{\xx}{{\bf x}}
\newcommand{\yy}{{\bf y}}
\newcommand{\zz}{{\bf z}}
\newcommand{\oomeg}{\mbox{\boldmath $\omega$}}
\newcommand{\res}{{\bf res}}
\newcommand{\cchi}{{\mbox{\raisebox{.4ex}{$\chi$}}}}
%\newcommand{\cchi}{{\cal X}}
%\newcommand{\cchi}{\mbox{\Large $\chi$}}

% Logical operators and symbols
\newcommand{\imply}{\Rightarrow}
\newcommand{\bimply}{\Leftrightarrow}
\newcommand{\union}{\cup}
\newcommand{\intersect}{\cap}
\newcommand{\boolor}{\vee}
\newcommand{\booland}{\wedge}
\newcommand{\boolimply}{\imply}
\newcommand{\boolbimply}{\bimply}
\newcommand{\boolnot}{\neg}
\newcommand{\boolsat}{\!\models}
\newcommand{\boolnsat}{\!\not\models}


\newcommand{\op}[1]{\mathrm{#1}}
\newcommand{\s}[1]{\ensuremath{\mathcal #1}}

% Properly styled differentiation and integration operators
\newcommand{\diff}[1]{\mathrm{\frac{d}{d\mathit{#1}}}}
\newcommand{\diffII}[1]{\mathrm{\frac{d^2}{d\mathit{#1}^2}}}
\newcommand{\intg}[4]{\int_{#3}^{#4} #1 \, \mathrm{d}#2}
\newcommand{\intgd}[4]{\int\!\!\!\!\int_{#4} #1 \, \mathrm{d}#2 \, \mathrm{d}#3}

% Large () brackets on different lines of an eqnarray environment
\newcommand{\Leftbrace}[1]{\left(\raisebox{0mm}[#1][#1]{}\right.}
\newcommand{\Rightbrace}[1]{\left.\raisebox{0mm}[#1][#1]{}\right)}

% Funky symobols for footnotes
\newcommand{\symbolfootnote}{\renewcommand{\thefootnote}{\fnsymbol{footnote}}}
% now add \symbolfootnote to the beginning of the document...

\newcommand{\normallinespacing}{\renewcommand{\baselinestretch}{1.5} \normalsize}
\newcommand{\mediumlinespacing}{\renewcommand{\baselinestretch}{1.2} \normalsize}
\newcommand{\narrowlinespacing}{\renewcommand{\baselinestretch}{1.0} \normalsize}
\newcommand{\bump}{\noalign{\vspace*{\doublerulesep}}}
\newcommand{\cell}{\multicolumn{1}{}{}}
\newcommand{\spann}{\mbox{span}}
\newcommand{\diagg}{\mbox{diag}}
\newcommand{\modd}{\mbox{mod}}
\newcommand{\minn}{\mbox{min}}
\newcommand{\andd}{\mbox{and}}
\newcommand{\forr}{\mbox{for}}
\newcommand{\EE}{\mbox{E}}

\newcommand{\deff}{\stackrel{\mathrm{def}}{=}}
\newcommand{\syncc}{~\stackrel{\textstyle \rhd\kern-0.57em\lhd}{\scriptstyle L}~}

\def\coop{\mbox{\large $\rhd\!\!\!\lhd$}}
\newcommand{\sync}[1]{\raisebox{-1.0ex}{$\;\stackrel{\coop}{\scriptscriptstyle
#1}\,$}}

\newtheorem{definition}{Definition}[chapter]
\newtheorem{theorem}{Theorem}[chapter]

\newcommand{\Figref}[1]{Figure~\ref{#1}}
\newcommand{\fig}[3]{
 \begin{figure}[!ht]
 \begin{center}
 \scalebox{#3}{\includegraphics{figs/#1.ps}}
 \vspace{-0.1in}
 \caption[ ]{\label{#1} #2}
 \end{center}
 \end{figure}
}

\newcommand{\figtwo}[8]{
 \begin{figure}
 \parbox[b]{#4 \textwidth}{
 \begin{center}
 \scalebox{#3}{\includegraphics{figs/#1.ps}}
 \vspace{-0.1in}
 \caption{\label{#1}#2}
 \end{center}
 }
 \hfill
 \parbox[b]{#8 \textwidth}{
 \begin{center}
 \scalebox{#7}{\includegraphics{figs/#5.ps}}
 \vspace{-0.1in}
 \caption{\label{#5}#6}
 \end{center}
 }
 \end{figure}
}